\documentclass[11pt]{article}

\usepackage{graphicx}
\usepackage{lingmacros}
\usepackage{verbatim}
\usepackage{color}

\usepackage[top=1in, left=1in]{geometry}

\definecolor{darkblue}{rgb}{0,0,0.8}
\definecolor{darkgreen}{rgb}{0,0.5,0}
\definecolor{darkred}{rgb}{0.8,0,0}
\definecolor{brown}{rgb}{0.5,0.3,0}

\newcommand{\tag}[2]{[\textbf{\color{darkblue}#1}$_{t#2}$]}
\newcommand{\othertag}[2]{[\textbf{\color{darkgreen}#1}$_{ot#2}$]}
\newcommand{\otherothertag}[2]{[\textbf{\color{darkred}#1}$_{oot#2}$]}
\newcommand{\yetanothertag}[2]{[\textbf{\color{brown}#1}$_{yat#2}$]}

\begin{document}

\title{Title\\
Subtitle}

\author{Author}

\date{\today}

\maketitle

\tableofcontents
\newpage

\section{Introduction} % (fold)
\label{sec:introduction}
Introducing \ldots
% section introduction (end)

\section{Extent Tags} % (fold)
\label{sec:extent_tag}

\subsection{Tag} % (fold)
\label{sub:tag}
How do you annotate a \textsc{tag}? Let me tell you \ldots
% subsection tag (end)

\subsubsection{Tag Extents} % (fold)
\label{ssub:tag_extents}

Example (\ref{ex:tag_examples}) demonstrates examples of \textsc{tag} extents:

\eenumsentence{
    \item This is an enumerated \tag{tag}{1} example.
    \label{ex:tag_command}
    \item You can give a tag an arbitrary \tag{id number}{9001}
    \label{ex:tag_id}
}\label{ex:tag_examples}

You can modify or expand the id wrapper commands to your liking and define your own colors as well. \texttt{lingstyle.sty} provides the \texttt{\textbackslash\-eenumsentence\{\ldots\}} environment for enumerated, multi-sentence examples and \texttt{\textbackslash\-enumsentence\{\ldots\}} for single examples.

If you \texttt{\textbackslash\-label\{ex:a\_label\}} your examples, you can refer to them (e.g., Example (\ref{ex:tag_command}) \& Example (\ref{ex:tag_id})). You can use the cleveref package if you want to have more control over references.
% subsubsection tag_extents (end)

\subsubsection{Tag Attributes} % (fold)
\label{ssub:tag_attributes}
Attributes for the \textsc{tag} tag are listed in Table \ref{tab:tag}.

\begin{table}[h]
\begin{center}
\begin{tabular}{|l|l|}
    \hline \textbf{Attribute}   & \textbf{Value}\\
    \hline \texttt{id}          & \texttt{t1}, \texttt{t2}, \texttt{t3}, \ldots\\
    \hline \texttt{attribute1}  & Takes a value\\
    \hline
\end{tabular}
\caption{\textsc{tag} Attributes}
\label{tab:tag}
\end{center}
\end{table}
% subsubsection tag_attributes (end)

\subsection{Other Tag} % (fold)
\label{sub:other_tag}
And, \textsc{othertag}? Well \ldots
% subsection other_tag (end)

\subsubsection{Other Tag Extents} % (fold)
\label{ssub:other_tag_extents}

% subsubsection other_tag_extents (end)

\subsubsection{Other Tag Attributes} % (fold)

\label{ssub:other_tag_attributes}
\begin{table}[h]
\begin{center}
\begin{tabular}{|l|l|}
    \hline \textbf{Attribute}   & \textbf{Value}\\
    \hline \texttt{id}          & \texttt{ot1}, \texttt{ot2}, \texttt{ot3}, \ldots\\
    \hline \texttt{attribute1}  & Takes a value\\
    \hline
\end{tabular}
\caption{\textsc{othertag} Attributes}
\label{tab:othertag}
\end{center}
\end{table}
% subsubsection other_tag_attributes (end)

% section extent_tags (end)

\section{Link Tags} % (fold)
\label{sec:link_tags}

\subsection{Other Other Tag} % (fold)
\label{sub:other_other_tag}
Let me tell you about \otherothertag{otherothertags}{1} \ldots

\subsubsection{Other Other Tag Attributes} % (fold)
\label{ssub:other_other_tag_attributes}

% subsubsection other_other_tag_attributes (end)

% subsection other_other_tag (end)

\subsection{Yet Another Tag} % (fold)
\label{sub:yet_another_tag}
Finally, let's discuss \yetanothertag{yetanothertag}{1} \ldots
% subsection yet_another_tag (end)

\subsubsection{Yet Another Tag Attributes} % (fold)
\label{ssub:yet_another_tag_attributes}

% subsubsection yet_another_tag_attributes (end)

% section link_tags (end)

\end{document}
