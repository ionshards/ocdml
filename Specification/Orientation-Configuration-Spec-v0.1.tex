\documentclass[11pt]{article}
\usepackage{hyperref}
    \hypersetup{colorlinks,breaklinks,
                urlcolor=[rgb]{0,0,0.8},
                linkcolor=[rgb]{0,0,0},
                citecolor=[rgb]{0,0,0}}
\usepackage{cleveref}
    \creflabelformat{enums}{(#2#1#3)}
        \crefname{enums}{example}{examples}
        \Crefname{enums}{Example}{Examples}
    \creflabelformat{enumsi}{(#2#1#3)}
        \crefname{enumsi}{example}{examples}
        \Crefname{enumsi}{Example}{Examples}
    \newcommand{\crefrangeconjunction}{~through~}
\usepackage{lingstyle}
\usepackage{graphicx}
\usepackage{color}

\usepackage[top=1in, left=1in]{geometry}

\definecolor{darkblue}{rgb}{0,0,0.8}
\definecolor{darkgreen}{rgb}{0,0.5,0}
\definecolor{darkred}{rgb}{0.8,0,0}

\newcommand{\entity}[2]{[\textbf{\color{darkblue}#1}$_{se#2}$]}
\newcommand{\signal}[2]{[\textbf{\color{darkred}#1}$_{os#2}$]}

\newcommand{\nameML}{Orientation Configuration Markup Language}
\newcommand{\ML}{OCML}

\newenvironment{attributes}
{
\begin{tabular}{|l|l|}
    \hline \textbf{Attribute} & \textbf{Value}\\
}
{   \hline
\end{tabular}
}

\newenvironment{note}
{\list{}
 {\setlength
  {\itemindent}
  {\listparindent}}
   \item[]\relax}
{\endlist}



\title{\nameML:\\
{\Large Version 0.1}}
\author{
    Cynthia Goodman\\
    \and
    Jasper Phillips\\
    \and
    Cory Massaro\\
    \and
    Zachary Yocum\\
}
\date{\today}


\begin{document}

\maketitle

\tableofcontents 
\newpage

\section{Introduction} % (fold)
\label{sec:introduction}

This document is a preliminary outline of \nameML~(\ML), a schema aimed at annotating spatial information, specifically intrinsic orientation of spatial entities, as encoded in natural language. This specification builds, in part, on ISO-Space\cite{pustejovsky_moszkowicz:2011}, an annotation standard being developed at Brandeis University for markup of spatial and spatio-temporal information. Additionally, we are evaluating the inventory of locative relations used by the Berkeley Transcription System\cite{slobin:1999} for American Sign Language.

\ML~anticipates an annotation task with the goal of identifying spatially relevant entities and categorizing their intrinsic spatial orientations. Our model presumes that there exist constraints on the intrinsic properties of spatial entities that humans rely on.\footnote{We do not suggest that these constraints are the same as those formalized in modern universal laws of physics or mathematics. Rather, we presume that these constraints belong to the set of common knowledge of human beings in line with the notion of na\"ive physics from Western philosophical and cognitive psychological traditions.} In \emph{The Design of Everyday Things}\cite{norman:2002}, cognitive scientist Donald A. Norman discusses physical objects that people interact with regularly in their daily lives in terms of the converse of such constraints, which he calls ``affordances''. So, in such terms, \ML~is intended to capture the intrinsic spatial affordances of spatial entities that are exploited by natural language.

Some examples of the kinds of spatial language which \ML~is intended to capture are listed in \cref{ex:intro_examples}.

\eenumsentence{
    \item I found a baby bird in the road at my work today.
    \label{ex:bird}
    \item \ldots an empty flask stood on the table beside him.
    \label{ex:flask}
    \item If I held the iPhone flat on its back, the way it would lay on a table \ldots
    \label{ex:iPhone}
}\label{ex:intro_examples}

For instance, in \cref{ex:bird}, the spatial entity \emph{road} can be categorized as a surface area whose intrinsic `top' affords `supporting'. In addition, the entity denoted by \emph{bird} can be categorized as a biped with an intrinsic `bottom' and affords `being supported'. Such affordances are exploited for the purposes of interpreting the relation introduced by the spatial preposition \emph{in}. In \cref{ex:flask}, it is the intrinsic `bottom' of the \emph{flask} which affords `standing' and the \emph{table}, much like the \emph{road} from the previous example, can be categorized as a surface whose `top' affords `supporting'. Also, assuming \emph{him} can be categorized as a humanoid biped, it is the intrinsic `left' or `right' side of \emph{him} that \emph{beside} selects for as the location for the \emph{table}. In \cref{ex:iPhone}, the \emph{iPhone} can be categorized as a slab-like volume with an intrinsic `back' which affords `laying'.
% section introduction (end)

\section{Goal} % (fold)
\label{sec:goal}
The goal of our schema is to adequately capture the intrinsic spatial orientation of spatial entities from natural language text. It may be possible that our schema can augment existing spatial annotation standards such as ISO-Space and may be useful for spatial reasoning and text-to-scene generation applications.
% section goal (end)

\section{Corpus Selection} % (fold)
\label{sec:corpus_selection}
Some corpora concerned with spatial positioning already exist in the form of scene descriptions for projects such as \href{http://imageclef.org/photodata}{ImageCLEF} and \href{http://lucky.cs.columbia.edu:2001/}{WordsEye}. These projects are likely to proffer useful data because the descriptive texts necessarily comprise spatial information. Further, and perhaps more linguistically natural, data can be gained from highly descriptive portions of literary texts. The easiest such portions to isolate and identify automatically would be stage directions and scene descriptions from dramatic pieces. Public domain literary texts can easily (and legally) be accessed via \href{http://www.gutenberg.org}{Project Gutenberg}; the Library of Congress's website (\href{loc.gov}{loc.gov}) and the Internet Archive (\href{archive.org}{archive.org}) also provide large literary corpora of public domain and Creative Commons-licensed works.
% section corpus_selection (end)

\section{Tag Types} % (fold)
\label{sec:tag_types}

This section enumerates the inventory of \ML~tag types.

\subsection{Spatial Entity} % (fold)
\label{sub:spatial_entity}

The \textsc{spatial\_entity} tag is used to capture participants in spatial relations. \ML~inherits the ISO-Space \textsc{spatial\_entity} tag which possesses \texttt{dimensionality}, \texttt{form}, \texttt{latLong}, \texttt{mod}, \texttt{countable}, \texttt{amount}, \texttt{quant}, and \texttt{scopes} attributes. \ML~extends the \textsc{spatial\_entity} tag with the following attributes: \texttt{top\_bottom}, \texttt{front\_back}, and \texttt{left\_right}. \Cref{tab:spatial_entity} enumerates the relevant ISO-Space attributes and \ML-specific attributes for the \textsc{spatial\_entity} tag.

\begin{table}[h]
\centering
\begin{attributes}
    \hline \texttt{id}                  & \texttt{se1}, \texttt{se2}, 
                                          \texttt{se3}, \ldots\\
    \hline \texttt{dimensionality}      & \textsc{point}, \textsc{line}, \textsc{area}, \textsc{volume}\\
    \hline \texttt{mod}                 & A spatially relevant modifier\\
    \hline \texttt{area\_type}          & \textsc{3-gon}, \textsc{4-gon}, \textsc{6-gon}, \textsc{8-gon}, \textsc{disc}, \textsc{annulus}\\
    \hline \texttt{volume\_type}        & \textsc{slab}, \textsc{cube}, \textsc{sphere}, \textsc{torus}, \textsc{cylinder}, \textsc{cone}, \textsc{biped}, \textsc{quadruped}\\
    \hline \texttt{left\_right}         & \textsc{intrinsic} or \textsc{relative}\\
    \hline \texttt{front\_back}         & \textsc{intrinsic} or \textsc{relative}\\
    \hline \texttt{top\_bottom}         & \textsc{intrinsic} or \textsc{relative}\\
\end{attributes}
\caption{Attributes for \textsc{spatial\_entity}}
\label{tab:spatial_entity}
\end{table}

\paragraph{\texttt{dimensionality}} % (fold)
\label{par:dimensionality}
This attribute designates the number of dimensions possessed by the entity. A value of \textsc{point} indicates a 0-dimensional entity consisting of a single point with no edges or surfaces. A value of \textsc{line} indicates a 1-dimensional entity with up to two bounding points with a single edge and no surfaces. A value of \textsc{area} indicates a 2-dimensional entity with some number of bounding edges, points and two surfaces. A value of \textsc{volume} indicates a 3-dimensional entity with at least one surface and possibly a number of bounding edges and points.
% paragraph dimensionality (end)

\paragraph{\texttt{area\_type}} % (fold)
\label{par:area_type}
This attribute is used to classify the entity based on a set of 2-dimensional geometric primitive shapes.
% paragraph area_type (end)

\paragraph{\texttt{volume\_type}} % (fold)
\label{par:volume_type}
This attribute is used to classify the entity based on a set of 3-dimensional geometric primitive shapes.
% paragraph volume_type (end)

\paragraph{\texttt{left\_right}} % (fold)
\label{par:left_right}
This attribute is a bit that indicates whether the entity possesses an intrinsic axis of orientation whose polar extremes are `left' and `right'. One heuristic which may help to determine whether a spatial entity possesses an intrinsic \texttt{left\_right} axis is the \emph{linear array test}. The \emph{linear array test} can be used by asking ``Could several of these entities be arranged in 1-dimensional array such that the `left' and `right' sides of each entity are all aligned with each other?'' If the answer is ``no'', then the \texttt{left\_right} attribute would be \textsc{relative}; if ``yes'', the value would be \textsc{intrinsic}.
% paragraph left_right (end)

\paragraph{\texttt{front\_back}} % (fold)
\label{par:front_back}
This attribute is similar to the \texttt{left\_right} attribute. A value of \textsc{intrinsic} indicates the entity possesses an intrinsic axis of orientation whose polar extremes are `front' and `back'. A value of \textsc{relative} indicates the opposite. The \emph{linear array test} heuristic can be modified for this case such that arrangement is `front-to-back'.
% paragraph front_back (end)

\paragraph{\texttt{top\_bottom}} % (fold)
\label{par:top_bottom}
This attribute is similar, again, to both \texttt{left\_right} and \texttt{front\_back}. A value of \textsc{intrinsic} indicates the entity possesses an intrinsic axis of orientation whose polar extremes are `top' and `bottom'. The \emph{linear array test} can be applied for this attribute as well to test if the entities can be stacked `top-to-bottom'.
% paragraph top_bottom (end)

\vspace{\parsep}

\Cref{ex:spatial_entities} lists examples of extents which would be appropriate candidates for the \textsc{spatial\_entity} tag:

\eenumsentence{
    \item \ldots a baby \entity{bird}{1} in the \entity{road}{2} at my work today.
    \label{ex:bird_road}
    \item \ldots an empty \entity{flask}{3} stood on the \entity{table}{4} beside \entity{him}{5}.
    \label{ex:flast_table}
    \item If \entity{I}{6} held the \entity{iPhone}{7} flat on its \entity{back}{8}, the way it would lay on a \entity{table}{9} \ldots
    \label{ex:iPhone_back}
}\label{ex:spatial_entities}
% subsection spatial_entity (end)

\subsection{Orientation Signal} % (fold)
\label{sub:orientation_signal}
The \textsc{orientation\_signal} tag is used to capture additional information about the orientation of a \textsc{spatial\_entity} that is not contributed by the lexical item associated with \textsc{spatial\_entity} extent itself. \Cref{tab:orientation_signal} enumerates the attributes for this tag type and \cref{ex:orientation_signals} lists examples of appropriate extents. We will be investigating further whether it is preferable to extend the ISO-Space \textsc{spatial\_signal} tag.

\begin{table}[h]
\centering
\begin{attributes}
    \hline \texttt{id}                  & \texttt{os1}, \texttt{os2}, 
                                          \texttt{os3}, \ldots\\
    \hline \texttt{orientation\_type}   & \textsc{longitudinal}, \textsc{latitudinal}, \textsc{vertical}, \textsc{other}\\
\end{attributes}
\caption{Attributes for \textsc{spatial\_entity}}
\label{tab:orientation_signal}
\end{table}

\eenumsentence{
    \item \ldots an empty flask \signal{stood}{1} on the table beside him.
    \label{ex:stood}
    \item If I held the iPhone \signal{flat}{2} on its back, the way it would \signal{lay}{3} on a table \ldots
    \label{ex:flat}
}\label{ex:orientation_signals}

% subsection orientation_signal (end)
% section tag_types (end)

\section{Link Types} % (fold)
\label{sec:link_types}

\subsection{Configuration Link} % (fold)
\label{sub:configuration_link}
The \textsc{configuration\_link} relates two \textsc{spatial\_entity} tags. The attributes are enumerated in \cref{tab:configuration_link}. This is a preliminary sketch and will likely be modified as our model continues to develop. We are weakly defining the \texttt{cluster} attribute here as we have not come up with a complete sense inventory at this time. The \texttt{coercion} attribute can be used to indicate that the dimensionality of a figure or ground is coerced to that of the other argument. The \texttt{direction} attribute is used to indicate the asymmetrical directionality of the relation in terms of figure-to-ground or ground-to-figure. 

\begin{table}[h]
\centering
\begin{attributes}
    \hline \texttt{id}                  & \texttt{cl1}, \texttt{cl2}, 
                                          \texttt{cl3}, \ldots\\
    \hline \texttt{cluster}             & A value from a sense inventory\\
    \hline \texttt{coercion}            & \textsc{figure}, \textsc{ground}, or \textsc{none}\\
    \hline \texttt{trigger}             & An ID of a \texttt{orientation\_signal} tag that triggers the link\\
    \hline \texttt{direction}           & \textsc{figure\_ground}, \textsc{ground\_figure}\\
\end{attributes}
\caption{Attributes for \textsc{configuration\_link}}
\label{tab:configuration_link}
\end{table}

% subsection configuration_link (end)
% section link_types (end)

\bibliographystyle{plain}
\bibliography{initial-OCML-spec}

\end{document}